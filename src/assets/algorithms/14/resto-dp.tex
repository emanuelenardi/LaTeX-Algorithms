%&../preamble

% arara: pdflatex: { synctex: no }
% arara: latexmk: { clean: partial }
\ifstandalone
\begin{document}
\begin{algorithm}[H]
\fi

\prototype{\restoDP{\Array{\Int} t, \Int n, \Int R}}{
	\params{t}[tagli disponibili]
	\params{n}[numero di monete]
	\params{R}[il resto da dare]

	\BlankLine
	\BlankLine
	\(DP\) \Assign \new \Array{\Int}[0][R]\;
	\(S\) \Assign \new \Array{\Int}[0][R]\;
	\(DP[0] \Assign 0\) \Comment*[h]{caso base}\;

	\BlankLine
	\tcp{Riempire la tabella \(DP\)}
	\From{\(i \Assign 1\) \DownTo \(R\)} {
		\(DP[i] \Assign +\infty\)\;

		\BlankLine
		\BlankLine
		\From(\Comment*[h]{Riempio la tabella}){\(j \Assign 1\) \DownTo \(n\)} {
			\If{\(t[j] \leqslant i\) \And \(DP[i-t[j]] + 1 < DP[i]\)}{
				\tcp{aggiorno il valore}

				\BlankLine
				\(DP[i] \Assign DP[i - t[j]] + 1\) \Comment*[h]{registro il valore}\;
				\(S[i] \Assign j\) \Comment*[h]{la moneta da utilizzare per risolvere il problema quando il taglio è \(i\)}\;
			}
		}
	}

	\BlankLine
	\While(\Comment*[h]{ho resto da dare}){\(R > 0\)}{
		\Print \(t[S[R]]\) \Comment*[h]{stampo la moneta}\;
		\(R \Assign R - t[S[R]]\) \Comment*[h]{decremento il resto}\;
	}
}

\ifstandalone
\end{algorithm}
\end{document}
\fi
