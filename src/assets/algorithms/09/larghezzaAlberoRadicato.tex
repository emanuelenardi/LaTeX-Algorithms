%&../preamble

\SetKwFunction{larghezzaAR}{larghezzaAlberoRadicato}

% arara: pdflatex: { synctex: no }
% arara: latexmk: { clean: partial }
\ifstandalone
\begin{document}
\begin{algorithm}[H]
\caption[Visita in profondità iterativa]{}
\fi

La procedura si ottiene semplicemente sostituendo una pila alla coda utilizzata nella visita \breadthFirstSearch.
\BlankLine
\tcp{effettua una visita in profondità iterativa}
\prototype{\larghezzaAR{\Graph G, \Node r}}{
	\Int \(max = 0\)\;
	\Queue \(S = \queueConstructor\)\;
	\(S.\queueInsert{r}\)\;

	\BlankLine
	\tcp{inizializzazione}
	\Array{\Bool} \(dist =\) \new \Array{\Bool}[1][G.\graphSize]\;
	\Array{\Bool} \(count =\) \new \Array{\Bool}[1][G.\graphSize]\;
	\ForEach{\(u \in G.\VV - \{r\}\)}{
		\(dist[u] = -1\)\;
		\(count[u] = 0\)\;
	}

	\BlankLine
	\(dist[r] = 0\) \Comment*[l]{la radice dista 0 da sé stessa}
	\While{\Not \(S.\queueEmpty\)}{
		\Node \(u = S.\queueRemove\) \Comment*[l]{estrai il nodo}

		\BlankLine
		\ForEach{\(v \in G.\adj{u}\)}{

			\BlankLine
			\If{\(dist[v] < 0\)}{
				\tcp{non ho ancora scoperto il nodo}

				\BlankLine
				\alert{\(dist[v] = dist[u] + 1\)} \Comment*[l]{imposta la distanza}
				\alert{\(\Increment{count[dist[v]]}\)} \Comment*[l]{ho scoperto un nuovo nodo}

				\BlankLine
				\(S.\queueInsert{v}\) \Comment*[l]{inserisci il nodo}
			}
		}
	}
}
\ifstandalone
\end{algorithm}
\end{document}
\fi
