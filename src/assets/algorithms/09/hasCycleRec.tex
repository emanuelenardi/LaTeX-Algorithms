%&../preamble

% arara: pdflatex: { synctex: no }
% arara: latexmk: { clean: partial }
\ifstandalone
\begin{document}
\begin{algorithm}[H]
\fi

\BlankLine
\tcp{restituisce \True se trova un ciclo}
\prototype{\Bool \hasCycleRec{\Graph \(G\), \Node \(u\), \Node \(p\), \Array{\Bool} \(visited\)}}{
	\params{G}[grafo esplorato]
	\params{u}[nodo da esaminare]
	\params{p}[nodo da cui provengo (padre)]
	\params{visited}[vettore dei nodi visitati]

	\BlankLine
	\(visited[u]\) \Assign \True \Comment*[l]{lo visito per la prima volta}

	\BlankLine
	\ForEach(\tcp*[h]{visito tutti i suoi vicini}){\(v \in G.\adj{u} - \{p\}\)}{
		\tcp{\(G.\adj{u} - \{p\}\): non considero il nodo da cui provengo (è un grafo orientato)}

		\BlankLine
		\uIf(\tcp*[h]{ho già visitato il nodo}){\(visited[v]\)}{
			\Return \True \Comment*[l]{ho trovato un ciclo}

			\BlankLine
			\tcp{altrimenti effettuo una visita ricorsiva sul nodo vicino \(v\)}
		}
		\ElseIf{\hasCycleRec{G, v, u, visited}}{
			\tcp{se una qualsiasi delle sottochiamate ritorna vero, allora ho trovato un ciclo}
			\Return \True\;
		}
	}

	\BlankLine
	\tcp{non ho trovato alcun ciclo}
	\Return \False\;
}

\ifstandalone
\end{algorithm}
\end{document}
\fi
