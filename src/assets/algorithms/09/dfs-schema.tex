%&../preamble

% arara: pdflatex: { synctex: no }
% arara: latexmk: { clean: partial }
\ifstandalone
\begin{document}
\begin{algorithm}[H]
\fi

\BlankLine
\tcp{classifica i lati di un grafo}
\prototype{\dfsSchema{\Graph \(G\), \Node \(u\), \Int {\sffamily\&}\(time\), \Array{\Int} \(dt\), \Array{\Int} \(ft\)}}{

	\params{time}[contatore]
	\params{dt}[tempo di scoperta]
	\params{ft}[tempo di fine]

	\BlankLine
	esamina il nodo \(u\) (caso \emph{pre}-visita)

	\BlankLine
	\Increment{time} \Comment*[l]{incremento il contatore}
	\(dt[u]\) \Assign \(time\) \Comment*[l]{lo memorizzo nel vettore di scoperta}

	\BlankLine
	\tcp{effettuo una visita in profondità}
	\ForEach{\(v \in G.\adj{u}\)}{

		\BlankLine
		\{ \textcolor{black}{esamina l'arco \((u,v)\) (qualsiasi)} \} \Comment*[l]{qui si sviluppa la logica dell'algoritmo}

		\BlankLine
		\uIf(\tcp*[h]{non ho ancora esaminato il nodo}){\(dt[v] \Equal 0\)}{
			\{ \textcolor{arcoAlbero}{esamina l'arco \((u,v)\) (albero)} \}\;

			\BlankLine
			\dfsSchema{G, v, time, dt, ft} \Comment*[l]{effettuo la chiamata ricorsiva}
			\BlankLine

		}
		\uElseIf{\(dt[u] > dt[v]\) \And \(ft[v] \Equal 0\)}{
			\tcp{se raggiungo un mio discendente e non ho ancora terminato la mia visita, allora ho trovato un arco \emph{all'indietro}}
			\{ \textcolor{arcoIndietro}{esamina l'arco \((u,v)\) (indietro)} \}
			\BlankLine

		}
		\uElseIf{\(dt[u] < dt[v]\) \And \(ft[v] \Neq 0\)}{
			\tcp{se raggiungo un mio discendente e ho terminato la mia visita, allora ho trovato un arco \emph{in avanti}}
			\{ \textcolor{arcoAvanti}{esamina l'arco \((u,v)\) (avanti)} \}
			\BlankLine

		}
		\Else{
			\tcp{l'ultimo caso rimanente}
			\{ \textcolor{arcoAttraversamento}{esamina l'arco \((u,v)\) (attraversamento)} \}
		}
	}


	\BlankLine
	\{ visita il nodo \(u\) (\emph{post}-visita) \}

	\BlankLine
	\Increment{time} \tcp{aggiorno il contatore}
	\(ft[u]\) \Assign \(time\) \Comment*[l]{lo memorizzo nel vettore di fine}
}

\ifstandalone
\end{algorithm}
\end{document}
\fi
