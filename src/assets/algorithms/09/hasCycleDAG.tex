%&../preamble

% arara: pdflatex: { synctex: no }
% arara: latexmk: { clean: partial }
\ifstandalone
\begin{document}
\begin{algorithm}[H]
\fi

\BlankLine
\tcp{applicabile solo ai DAG, in quanto non hanno archi all'indietro}
\prototype{\Bool \hasCycle{\Graph \(G\), \Node \(u\), \Int \(\&time\), \Array{\Int} \(dt\), \Array{\Int} \(ft\)}}{
	\tcp{\(u\): il primo nodo che viene visitato}

	\BlankLine
	\Increment{time} \Comment*[l]{aumento il contatore}
	\(dt[u]\) \Assign \(time\) \Comment*[l]{memorizzo il tempo di scoperta}

	\BlankLine
	\ForEach{\(v \in G.\adj{u}\)}{

		\BlankLine
		\uIf(\tcp*[h]{non ho ancora scoperto questo nodo}){\(dt[v] \Equal 0\)}{

			\BlankLine
			\tcp{effettuo una visita ricorsiva}
			\If{\hasCycle{G, v, time, dt, ft}}{
				\Return \True\;
			}
		}

		\BlankLine
		\tcp{logica dell'algoritmo}
		\ElseIf{\(dt[u] > dt[v]\) \And \(ft[v] \Equal 0\)}{
			\tcp{se raggiungo un mio discendente e non ho ancora terminato la mia visita, allora ho trovato un arco \emph{all'indietro} (un ciclo)}
			\Return \True\;
		}
	}

	\BlankLine
	\Increment{time} \Comment*[l]{aumento il contatore}
	\(ft[u]\) \Assign \(time\) \Comment*[l]{memorizzo il tempo di fine}

	\BlankLine
	\tcp{non ho trovato un ciclo}
	\Return \False\;
}

\ifstandalone
\end{algorithm}
\end{document}
\fi
