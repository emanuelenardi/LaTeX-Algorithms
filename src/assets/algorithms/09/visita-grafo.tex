%&../preamble

% arara: pdflatex: { synctex: no }
% arara: latexmk: { clean: partial }
\ifstandalone
\begin{document}
\begin{algorithm}[H]
\fi
\prototype{\visitaGrafo{\Graph G, \Node r}}{

	\BlankLine
	\Set \(S\) \Assign \setConstructor \Comment*[l]{insieme generico, da specificare (\Stack, \Queue)}
	\(S.\setInsert{r}\) \Comment*[l]{inserisco il nodo, da specificare}

	\BlankLine
	\tcp{ho visitato il nodo}
	\{ \alert{marca il nodo \(r\) come \enquote{scoperto}} \}

	\BlankLine
	\tcp{fintanto che l'insieme non è vuoto}
	\While{\(S.\setSize > 0\)}{

		\BlankLine
		\tcp{la politica di rimozione dipende dal problema da risolvere}
		\Node \(u \Assign S.\setRemove\)\;

		\BlankLine
		\{ \alert{esamina il nodo \(u\)} \}

		\BlankLine
		\ForEach{\(v \in G.\adj{u}\)}{

			\BlankLine
			\{ \alert{esamina l'arco \((u,v)\)} \}

			\BlankLine
			\If{\(v\) non è già stato scoperto}{

				\BlankLine
				\tcp{serve a non inserire il nodo più di una volta}
				\{ \alert{marca il nodo \(v\) come \enquote{scoperto}} \}\;

				\(S.\setInsert{v}\) \Comment*[l]{inserisce il nodo nell'insieme, da specificare}
			}
		}
	}
}
\ifstandalone
\end{algorithm}
\end{document}
\fi
