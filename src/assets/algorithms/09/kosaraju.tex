%&../preamble

% arara: pdflatex: { synctex: no }
% arara: latexmk: { clean: partial }
\ifstandalone
\begin{document}
\begin{algorithm}[H]
\caption[Algoritmo di Kosaraju]{}
\fi

\BlankLine
\tcp{effettua una visita in profondità del grafo, esaminando i nodi nell'ordine inverso di tempo di fine della prima visita.}
\prototype{\Array{\Int} \StrongConnectedComponents{\Graph \(G\)}}{
	\Stack \(S\) \Assign \topSort{G} \Comment*[l]{otteniamo i nodi in ordine decrescente di fine}
	\(G^T\) \Assign \transpose{G} \Comment*[l]{inverte il senso degli archi}
	\Return \ConnectedComponents{\(G^T\), \(S\)} \Comment*[h]{esegue una visita dfs sul grafo trasposto}
}

\BlankLine
\tcp{parte iterativa}
\prototype{\Array{\Int} \ConnectedComponents{\Graph G, \Stack S}}{
	\Array{\Int} \(id\) \Assign \new \Array{\Int}{1}{G.\stackSize}

	\BlankLine
	\ForEach{\(u \in G.\VV\)}{
		\(id[u]\) \Assign \(0\)
	}

	\BlankLine
	\Int \(counter\) \Assign \(0\)\;

	\While{\Not \(S.\stackEmpty\)}{
		\(u\) \Assign \(S\).\stackPop \Comment*[l]{estrazione in tempo inverso di tempo di fine}

		\BlankLine
		\If{\(id[u] \Equal 0\)}{
			\Increment{counter}\;
			\ccdfs{G, counter, n, id}\;
		}
	}

	\BlankLine
	\Return \(id\)
}

\BlankLine
\tcp{parte ricorsiva}
\prototype{\ccdfs{\Graph G, \Int counter, \Node n, \Array{\Int} id}}{
	\(id[u]\) \Assign \(counter\) \Comment*[l]{assegna il contatore a tutti i nodi che incontro}

	\BlankLine
	\ForEach{\(u \in G.\adj{u}\)}{
		\If{\(id[u] \Equal 0\)}{
			\ccdfs{G, counter, u, id}
		}
	}
}

\ifstandalone
\end{algorithm}
\end{document}
\fi
