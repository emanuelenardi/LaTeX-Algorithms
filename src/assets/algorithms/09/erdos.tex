%&../preamble

% arara: pdflatex: { synctex: no }
% arara: latexmk: { clean: partial }
\ifstandalone
\begin{document}
\begin{algorithm}[H]
\fi

\BlankLine
\tcp{il camminimo più breve fra due vertici viene memorizzato tramite il vettore dei padri \(p\)}
\prototype{\erdos{\Graph \(G\), \Node \(r\), \Array{\Int} \(erdos\), \Node \(parent\)}}{

	\BlankLine
	\tcp{struttura di supporto}
	\Queue \(S\) \Assign \queueConstructor \Comment*[l]{creo una pila}
	\(S.\queueInsert{r}\) \Comment*[l]{inserisco la radice}

	\BlankLine
	\tcp{inizializzazione}

	\lForEach(\tcp*[h]{nodi non ancora raggiunti}){\(u \in G.\VV - \{r\}\)}{
		\(erdos[u]\) \Assign \(\infty\)
	}
	\(erdos[r]\) \Assign \True \Comment*[l]{erdõs ha distanza 0 da se stesso}
	\(parent[r]\) \Assign \Nil \Comment*[l]{per la stampa del cammino}

	\BlankLine
	\tcp{visita del grafo}
	\While{\Not \(S.\setEmpty\)}{
		\Node \(u\) \Assign \(S\).\queueRemove\;

		\BlankLine
		\ForEach{\(u \in G.\adj{u}\)}{

			\BlankLine
			\{ \alert{esamina l'arco \((u,v)\)} \}

			\BlankLine
			\If{\(erdos[v] \Equal \infty\)}{
				\tcp{il nodo non è stato ancora stato scoperto}

				\BlankLine
				\(erdos[v]\) \Assign \(erdos[u] + 1\) \Comment*[l]{gli assegno un livello di erdõs+1}
				\(parent[v]\) \Assign \(u\) \Comment*[l]{memorizzo il padre del nodo attuale nel v.\ dei padri}
				\(S\).\queueInsert{v} \Comment*[l]{è la prima volta che lo raggiungo quindi lo metto in coda}
			}
		}
	}
}
\ifstandalone
\end{algorithm}
\end{document}
\fi
