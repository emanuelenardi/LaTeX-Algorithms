%&../preamble

% arara: pdflatex: { synctex: no }
% arara: latexmk: { clean: partial }
\ifstandalone
\begin{document}
\begin{algorithm}[H]
\fi

\BlankLine
\tcp{visitare tutti i nodi a distanza \(k\) prima di visitare i nodi a distanza \(k+1\)}
\prototype{\bfsProc{\Graph \(G\), \Node \(r\)}}{
	\Queue \(S\) \Assign \queueConstructor \Comment*[l]{creo una pila}
	\(S\).\queueInsert{\(r\)} \Comment*[l]{inserisco la radice}

	\BlankLine
	\tcp{inizializzazione}
	\Array{\Bool} \(visitato\) \Assign \Array{\Bool}[1][G.n] \Comment*[l]{della dimensione del no.\ di nodi}
	\lForEach(\tcp*[h]{devo ancora visitarli}){\(u \in G.\VV - \{r\}\)}{
		\(visitato[u]\) \Assign \False
	}
	\(visitato[r]\) \Assign \True \Comment*[l]{radice visitata}

	\BlankLine
	\tcp{visita del grafo}
	\While{\Not \(S\).\setEmpty}{
		\Node \(u\) \Assign \(S\).\queueRemove \Comment*[l]{rimuovo un nodo}

		\BlankLine
		\{ \alert{esamina il nodo \(u\)} \}

		\BlankLine
		\ForEach(\tcp*[h]{per ciascun nodo adiacente "v"}){\(v \in G.\adj{u}\)}{

			\BlankLine
			\{ \alert{esamina l'arco \((u,v)\)} \}

			\BlankLine
			\If(\tcp*[h]{se non ho ancora visitato "v"}){\Not \(visitato[v]\)}{

				\BlankLine
				\(visitato[v]\) \Assign \True \Comment*[l]{marcalo come visitato}
				\(S.\queueInsert\) \Comment*[l]{inseriscilo nella coda}
			}
		}
	}
}
\ifstandalone
\end{algorithm}
\end{document}
\fi
