%&../preamble

% arara: pdflatex: { synctex: no }
% arara: latexmk: { clean: partial }
\ifstandalone
\begin{document}
\begin{algorithm}[H]
\fi

\prototype{\Bool \enumeration{\datiProblema, \Item{} \(S\), \Int i, \datiParziali}}{
	\params{S}[vettore contenente le soluzioni parziali \(S[1][i]\)]
	\params{i}[indice della scelta corrente]

	\BlankLine
	\BlankLine
	\uIf{\isAdmissible{\datiProblema, \(S\), \(i\), \datiParziali}}{
		\tcp{è una soluzione ammissibile}
		\If{\processSolution{\datiProblema, \(S\), \(i\), \datiParziali}}{
			\tcp{vogliamo bloccare l'esecuzione alla prima soluzione trovata}
			\Return \True \tcp{trovata soluzione, restituisco \True}
		}
	}
	\uElseIf{\reject{\datiProblema, \(S\), \(i\), \datiParziali}}{

		\BlankLine
		\Return \False \tcp{impossibile trovare soluzioni, restituisco \False}
	}
	\Else{
		\Set \(C\) \Assign \getChoices{\datiProblema, \(S\), \(i\), \datiParziali} \tcp{l'insieme delle scelte possibili}

		\BlankLine
		\ForEach(\tcp*[h]{per ogni possibile scelta fra quelle possibili}){\(c \in C\)}{
			\(S[i]\) \Assign \(c\) \tcp{memorizzo la scelta parziale}

			\BlankLine
			\tcp{richiamo ricorsivamente l'algoritmo effettuando la scelta successiva}
			\If{\enumeration{\datiProblema, \(S\), \(i+1\), \datiParziali}}{
				\Return \True \tcp{trovata soluzione, restituisco \True}
			}
		}
	}
	\Return \False \tcp{nessuna soluzione, restituisco \False}
}

\ifstandalone
\end{algorithm}
\end{document}
\fi
