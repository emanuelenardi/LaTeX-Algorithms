%&../preamble

% arara: pdflatex: { synctex: no }
% arara: latexmk: { clean: partial }
\ifstandalone
\begin{document}
\begin{algorithm}[H]
\fi

\BlankLine
\tcp{ordina i sottovettori}
\prototype{\mergeSort{\Item{} \(A\), \Int \(primo\), \Int \(ultimo\)}}{

	\BlankLine
	\If(\Comment*[h]{devono esistere almeno due elementi}){\(primo < ultimo\)}{
		\Int \(mezzo \Assign \floor{\frac{primo + ultimo}{2}}\)\;
		\mergeSort{\(A\), \(primo\), \(mezzo\)}\;
		\mergeSort{\(A\), \(mezzo+1\), \(ultimo\)}\;
		\merge{\(A\), \(primo\), \(ultimo\), \(mezzo\)}\Comment*[l]{unisce le soluzioni}\;
	}
}

\ifstandalone
\end{algorithm}
\end{document}
\fi

% \BlankLine
% \BlankLine
% Equazione di ricorrenza:
% \[
% 	T =
% 	\begin{dcases}
% 		\Theta(1) & n = 1\\
% 		\T*{\nicefrac{n}{2}} + \T*{\nicefrac{n}{2}} + \Theta(n) & n > 1 \\
% 	\end{dcases}
% 	=
% 	\begin{dcases}
% 		c & n = 1\\
% 		2\T{\nicefrac{n}{2}} + dn & n > 1\\
% 	\end{dcases}
% \]
%
% \BlankLine
% Analisi per livelli:
% \[
% \Omicron \left( \sum_{i=0}^{k} \Ccancel{2^i} \frac{n}{\Ccancel{2^i}} \right) = \Omicron \left( \sum_{i=0}^{k} n \right) = \Omicron(k \cdot n) = \Omicron(n \log n)
% \]
% %
% Teorema dell'esperto:
%
% \begin{minipage}[t]{.4\linewidth}
% \begin{align*}
% 		\alpha &= \log_2 2 = 1 \\
% 		\beta  &= 1 \\
% 		\alpha &= \beta \\
% \end{align*}
% \end{minipage}
% \begin{minipage}[t]{.4\linewidth}
% \begin{align*}
% 	T &= \Omicron(n^{\alpha} \log n) \\
% 	  &= \Omicron(n \log n) \\
% \end{align*}
% \end{minipage}
