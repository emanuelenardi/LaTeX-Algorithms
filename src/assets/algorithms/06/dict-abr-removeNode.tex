%&../preamble

% arara: pdflatex: { synctex: no }
% arara: latexmk: { clean: partial }
\ifstandalone
\begin{document}
\begin{algorithm}[H]
\fi

\BlankLine
\tcp{RIMOZIONE DI UN NODO}
\prototype{\Tree \removeNode{\Tree \(T\), \Item \(k\)}}{
	% NOTE viene dichiarato ma non viene utilizzato
	% \Tree \(t\)\;

	\BlankLine
	\tcp{individuo il nodo da rimuovere}
	\Tree \(u \Assign\) \lookupNode{\(T, k\)}\;

	\BlankLine
	\tcp{se il nodo da rimuovere è presente nell'albero\dots}
	\If{\(u \neq \Nil\)}{
		\lnl{case:primo-caso}%
		\tcp{\dots e non ha figli}
		\uIf{\(u.\treeLeft \Equal \Nil\) \And \(u.\treeRight \Equal \Nil\)}{
		% \uIf(\tcp*[h]{nessun figlio}){\(u.\treeLeft \Equal \Nil\) \And \(u.\treeRight \Equal \Nil\)}{
			\If(\tcp*[h]{se esiste il padre}){\(u.\treeParent \Neq \Nil\)}{
				\shortcut{\(u.\treeParent, \Nil, k\)} \Comment*[l]{rimuovo il puntatore al figlio}
			}

			\BlankLine
			\tcp{rimuovo direttamente il nodo}
			\delete \(u\)\;
		}\lnlset{terzo-caso}{3}%
		\tcp{\dots ed ha due figli}
		\uIf{\(u.\treeLeft \neq \Nil\) \And \(u.\treeRight \neq \Nil\)}{
		% \uIf(\tcp*[h]{due figli}){\(u.\treeLeft \neq \Nil\) \And \(u.\treeRight \neq \Nil\)}{

			\BlankLine
			\Tree \(s\) \Assign \succNode \Comment*[l]{individuo il successore}

			\BlankLine
			\shortcut{\(s.\treeParent, s.\treeRight, s.\treeKey\)} \Comment*[l]{collego il sottoalbero destro}

			\BlankLine
			\tcp{copio il successore}
			\tcp{nella posizione del nodo rimosso}
			\(u.\treeKey \Assign s.\treeKey\)\;
			\(u.\treeValue \Assign s.\treeValue\)\;

			\BlankLine
			\tcp{rimuovo il successore}
			\delete \(s\)\;
		}\lnl{case:secondo-caso}%
		\tcp{\dots ed ha un solo figlio (sinistro)}
		\uIf{\(u.\treeLeft \neq \Nil\) \And \(u.\treeRight \Equal \Nil\)}{
		% \uIf(\tcp*[h]{solo un figlio (sinistro)}){\(u.\treeLeft \neq \Nil\) \And \(u.\treeRight \Equal \Nil\)}{

			\BlankLine
			\shortcut{\(u.\treeParent, u.\treeLeft, k\)} \Comment*[l]{collega il figlio al padre}

			\BlankLine
			\If(\tcp*[h]{se il padre non esiste}){\(u.\treeParent \Equal \Nil\)}{
				\(T \Equal u.\treeRight\) \Comment*[l]{il figlio diventa la radice}
			}
		}
		\tcp{\dots ed ha un solo figlio (sinistro)}
		\Else{
		% \Altrimenti(\tcp*[h]{solo un figlio (destro)}){

			\BlankLine
			\shortcut{\(u.\treeParent, u.\treeRight, k\)} \Comment*[l]{collega il figlio al padre}

			\BlankLine
			\If(\tcp*[h]{se il padre non esiste}){\(u.\treeParent \Equal \Nil\)}{
				\(T \Equal u.\treeRight\) \Comment*[l]{il figlio diventa la radice}
			}
		}
	}

	\BlankLine
	\tcp{restituisco la radice}
	\Return \(T\)\;
}

\ifstandalone
\end{algorithm}
\end{document}
\fi
