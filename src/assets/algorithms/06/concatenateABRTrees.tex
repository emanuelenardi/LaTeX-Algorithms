%&../preamble

\SetKwFunction{concatenateABRTrees}{concatenate}

% arara: pdflatex: { synctex: no }
% arara: latexmk: { clean: partial }
\ifstandalone
\begin{document}
\begin{algorithm}[H]
\fi
\begin{minipage}{\linewidth}

\BlankLine
\tcp{concatena due alberi di ricerca binaria}
\prototype{\concatenateABRTrees{\Tree \(T_1\), \Tree \(T_2\)}}{
	\Tree \(v = \maxFunction{\(T_1\)}\)\;
	\shortcut{v, \(T_2\), \(T_2\).\treeValue}\;
}

\BlankLine
\tcp{collega un nodo padre \(p\) ad un nodo figlio \(u\)}
\prototype{\shortcut{\Tree p, \Tree u, \Item x}}{
	\If{\(u \Neq \Nil\)}{
		\tcp{il nodo è stato cancellato}
		\(u.\treeParent = p\)\;
	}

	\BlankLine
	\If{\(p \Neq \Nil\)}{
		\eIf{\(x < p.\treeKey\)}{
			\(p.\treeLeft = u\)\;
		}{
			\(p.\treeRight = u\)\;
		}
	}
}

\end{minipage}
\ifstandalone
\end{algorithm}
\end{document}
\fi

% TODO da inglobare nel testo
Si cerca il massimo valore contenuto in \(T_1\) e lo si \enquote{attacca} alla radice di \(T_2\) come figlio destro di \(v\).
\concatenateABRTrees ha costo \(\Omicron(h)\), dove \(h\) è l'altezza massima fra i due alberi.
\shortcut ha costo \(\Omicron(1)\).

\paragraph{Analisi della complessità}
L'algoritmo ha complessità \(\Omicron(n)\), in quanto è il costo di una visita.
