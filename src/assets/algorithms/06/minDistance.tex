%&../preamble

% arara: pdflatex: { synctex: no }
% arara: latexmk: { clean: partial }
\ifstandalone
\begin{document}
\begin{algorithm}[H]
\fi
\begin{minipage}{.5\textwidth}

\BlankLine
% \tcp{determina la distanza \(d(x,y) = \abs{x-y}\) minima fra due elementi di un albero binario di ricerca}
\prototype{\Int \minDist{\Tree t}}{
	\Tree \(u = t.\minFunction\)\;
	\Int \(\varMin = +\infty\)\;
	\Int \(\varPrev = -\infty\)\;

	\BlankLine
	\While{\(u \Neq \Nil\)}{
		\If{\(u.\varValue - \varPrev < \varMin\)}{
			\(\varMin = u.\varValue - \varPrev\)\;
		}
		\(\varPrev = u.\varValue\)\;
		\(u = u.\succNode\)\;
	}
}
\end{minipage}%
\begin{minipage}{.5\textwidth}

\ifFigureOfAlgo
\vfill
\begin{center}
	\begin{forest} circled, math tree
		[6
			[4[1][3]]
			[9[,phantom][,phantom]]
		]
	\end{forest}
\end{center}
\vfill\null
% {\raggedright
% Si analizza l'albero seguendo l'ordine crescente (in-visita);
% per ogni numero incontrato, si calcola la distanza con il valore precedente e la si confronta con il minimo trovato finora.}% (opportunamente inizializzata a \(+\infty\)).
% % Il costo è pari al costo di visita dell'albero, ovvero \(\Omicron(n)\), dove \(n\) è il numero di nodi.
\fi

\end{minipage}
\ifstandalone
\end{algorithm}
\end{document}
\fi
