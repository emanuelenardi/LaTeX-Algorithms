%&../preamble

% arara: pdflatex: { synctex: no }
% arara: latexmk: { clean: partial }
\ifstandalone
\begin{document}
\begin{algorithm}[H]
\fi

\BlankLine
\prototype{\Int \dictLookup{\Item \(k\)}}{
	\Tree \(t\) \Assign \lookupNode{\(tree, k\)}\;

	\BlankLine
	\eIf{\(t \Neq \Nil\)}{
		\Return \(t\).\treeValue\;
	}{
		\Return \Nil\;
	}
}

\BlankLine
% \tcp{Restituisce il nodo dell'albero \(t\)}
% \tcp{che contiene la chiave \(k\),}
% \tcp{se presente, \Nil altrimenti}
\tcp{RICERCA DI UN NODO, iterativa}
\prototype{\Tree \lookupNode{\Tree \(T\), \Item \(k\)}}{
	\Tree \(u \Assign T\) \Comment*[l]{parto dalla radice}

	\BlankLine
	\While{\(u \Neq \Nil\) \And \(u.\treeKey \Neq k\)}{
		\(u\) \Assign \iif{\(k < u.\treeKey, u.\treeLeft, u.\treeRight\)}\;
	}
}

\BlankLine
\tcp{RICERCA DI UN NODO, ricorsiva}
\prototype{\Tree \lookupNode{\Tree \(T\), \Item \(k\)}}{
	\eIf{\(T \Equal \Nil\) \Or \(T.\treeKey \Equal k\)}{
		\Return \(T\)\;
	}{
		\Return \lookupNode{\iif{\(k < u.\treeKey, u.\treeLeft, u.\treeRight\)}, \(k\)}\;
	}
}
\BlankLine

\ifstandalone
\end{algorithm}
\end{document}
\fi
