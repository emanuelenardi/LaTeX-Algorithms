%&../settings/preamble.main

\ifsubfile
\pagestyle{plain}
\setcounter{chapter}{7}

% arara: pdflatex: { options: ["--output-directory=../build"], draft: yes, synctex: no }
% arara: pdflatex: { options: ["--output-directory=../build"], synctex: no }
\begin{document}
\fi
\chapter{Insiemi}

Possono essere implementati con molte delle strutture dati viste fin'ora.
Ognuna delle quali rappresenta valntaggi e svantaggi.

\section{Realizzazione con vettori booleani}

L'insieme viene rappresentato attraverso un vettore booleano di \(m\) elementi.
Il quale è notevolmente semplice da implementare ed estremamente efficiente verificare se un elemento appartiene all'insieme.
Sfortunatamente la memoria occupata è \(\Omicron(m)\), indipendentemente dalle dimensioni effettive, inoltre alcune operazioni dipendono dalla memoria utilizzata per memorizzare questi oggetti, piuttosto che dal numero di oggetti effettivamente memorizzati, il che porta ad una complessità di queste operazione di \(\Omicron(m)\).

\begin{algorithm}[H]
	\caption[Struttura dati insieme implementata come Vettore Booleano]
	        {Struttura dati \textsc{Set} implementata come Vettore Booleano}
	%&../preamble

% arara: pdflatex: { synctex: no }
% arara: latexmk: { clean: partial }
\ifstandalone
\begin{document}
\begin{algorithm}[H]
% \caption{Struttura dati \textsc{Set} implementata come Vettore Booleano}
% \the\columnsep
% \setlength{\columnsep}{-3.5cm}
% \the\columnsep
% \begin{multicols}{2}
\fi
% Una struttura dati \emph{dinamica}, \emph{non lineare} che memorizza una \emph{collezione non ordinata di elementi} senza valori ripetuti.

\begin{minipage}[t]{.4\textwidth}
\BlankLine
\Array{\Bool} \(V\)\;
\Int \(size\)\;
\Int \(dim\)\;

\BlankLine
\prototype{\Set \setConstructor{\Int m}}{
	\Set \(t \Assign\) \new \Set\;
	\(t.\varSize \Assign 0\)\;
	\(t.\varCapacity \Assign m\)\;
	\(t.V \Assign [\False] * m\)\;

	\BlankLine
	\Return \(t\)\;
}

\BlankLine
\prototype{\Set \setContains{\Int \(x\)}}{

	\BlankLine
	\eIf{\(1 \leqslant x \leqslant \varCapacity\)}{
		\Return \(V[x]\)\;
	}{
		\Return \(\False\)\;
	}
}

\BlankLine
\prototype{\Int \setSize}{
	\Return \(\varSize\)\;
}

\BlankLine
\prototype{\setInsert{\Int \(x\)}}{
	\If{\(1 \leqslant x \leqslant \varCapacity\)}{
		\If{\Not \(V[x]\)}{
			\(\Increment{\varSize}\)\;
			\(V[x] \Assign \True\)\;
		}
	}
}

\BlankLine
\prototype{\setRemove{\Int \(x\)}}{
	\If{\(1 \leqslant x \leqslant \varCapacity\)}{
		\If{\(V[x]\)}{
			\(\Decrement{\varSize}\)\;
			\(V[x] \Assign \False\)\;
		}
	}
}

\end{minipage}%
\begin{minipage}[t]{.5\textwidth}

\BlankLine
\prototype{\Set \setUnion{\Set \(A\), \Set \(B\)}}{
	\tcp{crea un insieme della capacità max}
	\Set \(C \Assign\) \setConstructor{\maxFunction{\(A.\varCapacity, A.\varCapacity\)}}\;

	\BlankLine
	\tcp{inserisci gli elementi di \(A\)}
	\From{\(i \Assign 1\) \DownTo \(A.\varCapacity\)}{
		\If{\(A.\setContains{i}\)}{
			\(C.\setInsert{i}\)\;
		}
	}

	\BlankLine
	\tcp{inserisci gli elementi di \(B\)}
	\From{\(i \Assign 1\) \DownTo \(B.\varCapacity\)}{
		\If{\(A.\setContains{i}\)}{
			\(C.\setInsert{i}\)\;
		}
	}
}

\BlankLine
\prototype{\Set \setIntersection{\Set \(A\), \Set \(B\)}}{
	\tcp{crea un insieme della capacità min}
	\Set \(C \Assign\) \setConstructor{\(\minFunction{A.\varCapacity, A.\varCapacity}\)}\;

	\BlankLine
	\From{\(i \Assign 1\) \DownTo \(\minFunction{A.\varCapacity, A.\varCapacity}\)}{
		\tcp{se è contenuto in entambi}
		\If{\(A.\setContains{i}\) \And \(B.\setContains{i}\)}{
			\(C.\setInsert{i}\) \Comment*[l]{aggiungilo}
		}
	}
}

\BlankLine
\prototype{\Set \setDifference{\Set \(A\), \Set \(B\)}}{
	\Set \(C \Assign\) \setConstructor{\(A.\varCapacity\)}\;

	\BlankLine
	\From{\(i \Assign 1\) \DownTo \(A.\varCapacity\)}{
		\tcp{se è contenuto \(A\) e non in \(B\)}
		\If{\(A.\setContains{i}\) \And \Not \(B.\setContains{i}\)}{
			\(C.\setInsert{i}\) \Comment*[l]{aggiungilo}
		}
	}
}

% \vspace{35pt}
% \vphantom{0pt}

\end{minipage}
\ifstandalone
% \end{multicols}
% \the\columnsep
% \setlength{\columnsep}{10pt}
% \the\columnsep
\end{algorithm}
\end{document}
\fi

\end{algorithm}

\clearpage
\subsection{Implementazioni nei linguaggi di programmazione}

Esistono alcune implementazioni nei linguaggi attualmente utilizzati.
In Java esiste la struttura dati \texttt{BitSet} i cui meotodi sono illustrati nella tabella~\ref{tab:java-set-implementation}.
Mentre in \texttt{C++ STL} esistono due implementazioni \texttt{std::bitset} e \texttt{vector<bool>}:
\begin{itemize}
	\item \texttt{bitset} è una struttura dati con dimensione fissata nel template al momento della compilazione;
	\item \texttt{vector<bool>} è una specializzazione di \texttt{vector} per ottimizzare la memorizzazione, ha dimensione dinamica.
\end{itemize}

\begin{table}[ht]
	\centering
	\caption{Implementazione \texttt{java.util.BitSet}}%
	\label{tab:java-set-implementation}
	\begin{tabular}{@{} l >{\ttfamily}l @{}}
	\toprule
		Operazione & \normalfont{Metodo} \\
	\midrule
		\setContains & boolean get(int i) \\
		\setSize & int cardinality() \\
		\setInsert & void set(int i) \\
		\setRemove & void clear(int i) \\
		\setUnion & void and(BitSet set) \\
		\setIntersection & void or(BitSet set) \\
		% \setDifference & \\
	\bottomrule
	\end{tabular}
\end{table}

\vspace{-10pt}
\section{Realizzazione con vettore non ordinato}

\begin{algorithm}[H]
	\caption[Struttura dati insieme implementata come vettore non ordinato]
	        {Struttura dati \textsc{Set} implementata come vettore non ordinato}
	%&../preamble

% arara: pdflatex: { synctex: no }
% arara: latexmk: { clean: partial }
\ifstandalone
\begin{document}
\begin{algorithm}[H]
% \caption{Struttura dati \textsc{Set} implementata come vettore non ordinato}
\fi

\prototype{\Set \setDifference{\Set \(A\), \Set \(B\)}}{
	\Set \(C \Assign\) \setConstructor \Comment*[l]{non ha bisogno della dimensione}

	\BlankLine
	\From{\(s \in A\)}{
		\tcp{se non è contenuto in \(B\)}
		\If{\Not \(B.\setContains{s}\)}{
			\(C.\setInsert{s}\) \Comment*[l]{aggiungilo}
		}
	}
}

\ifstandalone
\end{algorithm}
\end{document}
\fi

\end{algorithm}

\paragraph{Costo delle operazioni}
Le operazioni di ricerca, inserimento e cancellazione costano \(\Omicron(n)\), le operazioni di inserimento (assumendo che non esista l'elemento) costano \(\Omicron(1)\), le operazioni di unione, intersezione e differenza \(\Omicron(nm)\).

\clearpage
\section{Realizzazione vettore ordinato}

\begin{algorithm}[H]
	\caption*{Struttura dati \textsc{Set} implementata come vettore ordinato}
	%&../preamble

% arara: pdflatex: { synctex: no }
% arara: latexmk: { clean: partial }
\ifstandalone
\begin{document}
\begin{algorithm}[H]
% \caption{Struttura dati \textsc{Set} implementata come vettore ordinato}
\fi

\prototype{\Set \setIntersection{\List \(A\), \List \(B\)}}{

	\BlankLine
	\List \(C\) \Assign \setConstructor \Comment*[l]{non ha bisogno della dimensione}

	\tcp{creo puntatori alle liste}
	\Pos \(p\) \Assign \(A\).\listHead\;
	\Pos \(q\) \Assign \(B\).\listHead\;

	\BlankLine
	\While{\Not \(A.\listEnd{p}\) \And \(B.\listEnd{q}\)}{

		\BlankLine
		\uIf(\tcp*[h]{se gli elementi coincidono}){\(A.\listRead{p} \Equal B.\listRead{q}\)}{
			\(C\).\listInsert{\(C\).\listTail, \(A\).\listRead{\(p\)}} \Comment*[l]{inseriscilo nell'intersezione}
			\tcp{scorri i puntatori}
			\(p\) \Assign \(A\).\listSucc{\(p\)}\;
			\(q\) \Assign \(B\).\listSucc{\(q\)}\;
		}
		\uElseIf{\(A.\listRead{p} < B.\listRead{q}\)}{
			\(p\) \Assign \(A\).\listSucc{\(p\)} \Comment*[l]{scorro puntatore di \(A\)}
		}
		\Else{
			\(q\) \Assign \(B\).\listSucc{\(q\)} \Comment*[l]{scorro puntatore di \(B\)}
		}
	}
}

\ifstandalone
\end{algorithm}
\end{document}
\fi

\end{algorithm}

\paragraph{Costo delle operazioni}
Le operazioni di ricerca costano \(\Omicron(n)\) con le liste e \(\Omicron(\log n)\) con i vettori, le operazioni di inserimento e cancellazione costano \(\Omicron(n)\), le operazioni di unione, intersezione e differenza \(\Omicron(n)\).

\section{Reality Check}

In realtà si utilizzano strutture dati complesse che permettono di ottimizzare le performance.
Se abbiamo bisogno dell'ordinamento si utilizzano alberi bilanciati, mentre se abbiamo bisogno di sapere smeplicemente se un elemento è contenuto o meno si utilizzano le tabelle hash.

\medskip
Per le operazioni di ricerca, inserimento e cancellazione negli alberi bilanciati hanno una complessità di \(\Omicron(\log n)\), mentre nelle tabelle hash di \(\Omicron(1)\).

\medskip
Le implementazioni più diffuse degli alberi bilanciati sono \texttt{TreeSet} in Java, \texttt{OrderedSet} in Python e \texttt{set} in \texttt{C++}, mentre le implementazioni più diffuse delle tabelle hash sono \texttt{HashSet} in Java, \texttt{set} in Python e \texttt{unordered\_set} in \texttt{C++}.

\begin{table}[!ht]
	\centering
	\caption[Implementazioni e relative complessità delle operazioni sugli insiemi]{Implementazioni e relative complessità delle operazioni\\\(m \equiv\) dimensione del vettore o della tabella hash}%
	\label{tab:set-resume}
	\begin{tabular}{@{} *8{l} @{}}
	\toprule
		& \makecell[l]{\setContains\\\lookup} & \setInsert & \setRemove & \makecell[l]{\minFunction\\\maxFunction} & \makecell[l]{\textsf{foreach}\\(memoria)} & Ordine \\
	\cmidrule{2-7}
		Vettore booleano & \(\Omicron(1)\) & \(\Omicron(1)\) & \(\Omicron(1)\) & \(\Omicron(m)\) & \(\Omicron(m)\) & Sì\\
	\addlinespace
		Lista non ordinata & \(\Omicron(n)\) & \(\Omicron(n)\) & \(\Omicron(n)\) & \(\Omicron(n)\) & \(\Omicron(n)\) & No\\
	\addlinespace
		Lista ordinata & \(\Omicron(n)\) & \(\Omicron(n)\) & \(\Omicron(n)\) & \(\Omicron(1)\) & \(\Omicron(n)\) & Sì\\
	\addlinespace
		Vettore ordinato & \(\Omicron(\log n)\) & \(\Omicron(n)\) & \(\Omicron(n)\) & \(\Omicron(1)\) & \(\Omicron(n)\) & Sì\\
	\addlinespace
		Alberi bilanciati & \(\Omicron(\log n)\) & \(\Omicron(\log n)\) & \(\Omicron(\log n)\) & \(\Omicron(\log n)\) & \(\Omicron(n)\) & Sì\\
	\addlinespace
		Hash (mem.\ interna) & \(\Omicron(1)\) & \(\Omicron(1)\) & \(\Omicron(1)\) & \(\Omicron(m)\) & \(\Omicron(n)\) & No\\
	\addlinespace
		Hash (mem.\ esterna) & \(\Omicron(1)\) & \(\Omicron(1)\) & \(\Omicron(1)\) & \(\Omicron(m+n)\) & \(\Omicron(m+n)\) & No\\
	\bottomrule
	\end{tabular}
\end{table}

\ifsubfile
\end{document}
\fi
