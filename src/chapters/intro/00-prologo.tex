\chapter*{Prologo}
\addcontentsline{toc}{chapter}{Prologo}

Caro studente,

ho lavorato a questi appunti cercando di offrirti l'esperienza di lettura e di comprensione migliore possibile, rileggendo più volte, correggendo errori e facendole revisionare da alcuni tuoi compagni di corso che mi hanno aiutato a portarti un lavoro corretto e revisionato.

Ho cercato di rispettare la seguente affermazione tratta da \enquote{Pensare in Python, Come pensare da Informatico}:
\begin{quote}
I commenti più utili sono quelli che documentano caratteristiche del codice di non immediata comprensione.
\`{E} ragionevole supporre che chi legge il codice possa capire \emph{cosa} esso faccia; è più utile spiegare \emph{perché}.
\end{quote}
Questa affermazione è sicuramente vera per coloro che hanno già delle forti basi di programmazione, per mia sfortuna e per fortuna per voi, questo non era il mio caso, quindi ogni algoritmo è stato analizzato, scomposto e svicerato di ogni sua componente fino ad arrivare al suo principio.
Non ho dato nessun preconcetto per scontato e tutti i passaggi, matematici e logici, sono stati resi espliciti in ogni momento possibile.

I seguenti appunti sono una traccia delle lezioni di Algoritmi e Strutture Dati, il mio consiglio per affrontare la materia è quella di leggere prima l'argomento e andare a lezione con le idee ben chiare, in questo modo ti verrà spontaneo fare domande su ciò che non ti è chiaro o che eventualmente non hai capito avendo già toccato quell'argomento da solo.
Non aver paura di far domande! Il professore è super disponibile ed è uno dei migliori del nostro ateneo, non si farà problemi a risponderti e, anche se la domanda sarà banale, è molto probabile che tu non sia l'unico ad avere quel dubbio.
Probabilmente farai un favore a qualcuno che è più timido di te!
E, cosa molto più importante, interagendo trarrai molto più beneficio dalla lezione in quanto porterai a casa un'esperienza individuale e ti sarà più semplice ricordarti quale fosse il tuo dubbio e la relativa risposta in sede d'esame.

Dall'anno accademico 2018/2019 il corso di Algoritmi è diventato corso annuale diviso in due parti, questo ha dato a molti studenti la possibilità di avere più tempo per interiorizzare le molte nozioni richieste per una comprensione piena degli argomenti trattati.
Nel mese di marzo quando riprenderai la materia in mano (che tu abbia affrontato il primo parziale o meno) è il caso che tu rilegga gli argomenti trattati nel primo semestre.

Se trovi errori di qualsiasi natura non indugiare a segnalarmeli, li correggerò il prima possibile e aggiornerò l'errata corrige, inoltre (nel caso ti facesse piacere) verrai menzionato vicino all'errore segnalato.
Una versione pubblica e aggiornata dell'errata corrige sarà presente alla pagina \enquote{\href{https://github.com/emanuelenardi/latex-algorithms/wiki/Errata-corrige}{Errata corrige}} nella Wiki del progetto github \href{https://github.com/emanuelenardi/latex-algorithms/wiki/Errata-corrige}{\texttt{github.com/emanuelenardi/latex-algorithms}}.

Vorrei ringraziare Francesco Bozzo, Samuele Conti e Filippo Frezza per aver revisionato questa dispensa, per la considerevole pazienza e meticolosità delle correzioni.
Gli errori rimanenti sono, ovviamente, interamente miei.
Le carenze di questa dispensa sarebbero considerevolmente maggiori e più numerose se non fosse stato per la loro assistenza.

Come ultima battuta vorrei ringraziare tutti coloro che mi hanno sostenuto in questi mesi per la stesura di questo testo e a coloro senza i quali sarebbe rimasti dei semplici appunti.

Ti auguro buono studio e di passare l'esame a pieni voti!

Emanuele
