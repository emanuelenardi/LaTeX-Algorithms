%&../settings/preamble.main

\ifsubfile
\pagestyle{plain}
\setcounter{chapter}{16}

% arara: pdflatex: { options: ["--output-directory=../build"], draft: yes, synctex: no }
% arara: pdflatex: { options: ["--output-directory=../build"], synctex: no }
\begin{document}
\fi
\section{Algoritmi probabilistici}

\begin{idea}
Se non sapete cosa fare fate una scelta casuale.
\end{idea}

\begin{note}
Il calcolo delle probabilità è applicato ai dati di output, non ai dati di input.
\end{note}

\section{Algoritmi Montecarlo}

\begin{idea}
Sono algoritmi in cui la correttezza è probabilistica.
\end{idea}

\texttt{MATERIALE MANCANTE}

\subsection{Test di primalità}

\begin{theorem*}[Piccolo teorema di fermat]
Per ogni numero primo \(n\) e ogni \(b \in [2, \dots, n-1] \colon b^{n-1} \bmod n = 1\).
\end{theorem*}

\begin{algorithm}[H]
\caption{Test di primalità}

\prototype{\isPrime{\Int \(n\)}}{

	\BlankLine
	\(b \Assign \random{2, \(n-1\)}\)\;

	\BlankLine
	\If{\(b^{n-1} \bmod n \Neq 1\)}{

		\BlankLine
		\tcp{non può essere un numero primo}
		\Return \False
	}

	\BlankLine
	\tcp{è un numero primo}
	\Return \True
}

\end{algorithm}

\section{Algoritmi Las Vegas}

\begin{idea}
Sono algoritmi corretti, in cui il funzionamento è probabilistico.
\end{idea}

\texttt{MATERIALE MANCANTE}

\subsection{Statistiche d'ordine}

\texttt{MATERIALE MANCANTE}

\begin{algorithm}[H]
\caption{inserisci didascalia}

\prototype{\heapSelect{\Item{} A, \Int n, \Int k}}{

	\heapBuild{A}\;

	\For{\(i=1\) \DownTo \(k-1\)}{
		\heapDeleteMin{A, n}
	}

	\tcp{restituisce il \(k\)-esimo valore}
	\Return \heapDeleteMin{A, n}
}

\end{algorithm}



\paragraph{Analisi della complessità}
Non va bene per \(k = \nicefrac{n}{2}\), perchè viene \(\Omicron(n + \nicefrac{n}{2} \log n) = \Omicron(n \log n)\)

Cambiamo tattica.
Utilizziamo una tecnica dividi-et-impera simile al \quickSort.
Ma, a differenza di quest'ultimo, non è necessario cercare in entrambe le partizioni.

\begin{algorithm}[H]
\caption{inserisci didascalia}

\prototype{\selection{\Item{} A, \Int start, \Int end, \Int k}}{

	\BlankLine
	\eIf{\(start \Equal end\)}{
		\tcp{caso base: ho un solo elemento}
		\Return \(A[start]\)
	}{

		\BlankLine
		\tcp{calcolo indici}
		\Int \(j = \pivot{A, start, end}\) \Comment*[l]{perno}
		\Int \(q = j - start +1\) \Comment*[l]{mediana}

		\BlankLine
		\uIf{\(k \Equal q\)}{
			\tcp{ho trovato il mio elemento}
			\Return \(A[j]\)
		}\uElseIf{\(k < q\)}{
			\tcp{cerco a sinistra}
			\Return \selection{A, start, \(j-1\), k}
		}
		\Else{
			\tcp{cerco a destra}
			\Return \selection{A, \(j+1\), end, \(k-q\)}
		}
	}
}

\end{algorithm}

% TODO figura esplicativa

\paragraph{Analisi della complessità}

Effettuiamo un'analisi simile a quella fatta per il \quickSort.

\subparagraph{Caso pessimo}

Il caso pessimo è lo stesso del \quickSort (che è il caso in cui il vettore sia già ordinato).
In quanto dividiamo il vettore con 0 elementi a sinistra ed \(n\) elementi a destra della partizione.
\begin{align*}
T(n) &=
	\begin{dcases}
	1			& n \leqslant 1 \\
	T(n-1) + n	& n > 1 \\
	\end{dcases} \\
T(n) &= \Omicron(n^2)
\end{align*}

Quindi nel caso pessimo questo algoritmo risulta peggiore di ordinare il vettore (\(\Omicron(n \log n)\)) e di andare a fare la ricerca.

\subparagraph{Caso ottimo}

Nel caso ottimo ogni volta divido esattamente a metà
\begin{align*}
	T(n) &=
	\begin{cases}
		1						& n \leqslant 1 \\
		T(\nicefrac{n}{2}) + n	& n > 1 \\
	\end{cases} \\
	T(n) &= O(n)
\end{align*}

\subparagraph{Caso medio}

Assumiamo che \pivot restituisca con la stessa probabilità una qualsiasi posizione \(j\) del vettore \(A\):
\[\begin{WithArrows}[displaystyle]
T(n) &= n-1 + \frac{1}{n}\sum_{q=1}^{n} T \left( \max\{q-1, n-q\} \right) \Arrow{per \(n > 1\)} \\
     &\leqslant n-1 + \frac{2}{n} \sum_{ q =  \floor{\nicefrac{n}{2}} }^{n-1} T(q)
\end{WithArrows}\]

\(\frac{1}{n}\) la media su tutti i possibili valori di \(q\)
\(\sum_{q=1}^{n}\) rappresenta la dimensionalità di \(j\), \(q-1\) gli elementi a sinistra, \(n-q\) a destra rispettivamente.

\(\exists c > 0, \exists m \geqslant 0: T(n) \leqslant cn, \forall n \geqslant m\)

\[\begin{WithArrows}[displaystyle]
T\left(n\right) &\leqslant n - 1 + \frac{2c}{n} \sum_{ q=\floor{n/2} }^{n-1} q \Arrow{divido la sommatoria a metà} \\
&\leqslant n + \frac{2c}{n} \left( \sum_{q=1}^{n-1} q - \sum_{q=1}^{\floor{n/2}-1} q \right)  \Arrow{algebra} \\
&= n + \frac{2c}{n}\left(\frac{n\left(n- 1\right)}{2} - \frac{ \floor{n/2} (\floor{n/2} - 1)}{2} \right) \Arrow{semplifico} \\
&\leqslant n + \frac{c}{n} \big( n(n- 1) - (n/2+1)(n/2) \big) \Arrow{eseguo i calcoli} \\
&= n + c\left(n - 1\right) - \left(c/2\right)\left(n/2+ 1\right) \Arrow{eseguo i calcoli} \\
&= n + cn - c - cn/4 - c/2 \Arrow{la mia disequazione risulta vera} \\
&= cn \left( \frac{1}{c}+ \frac{3}{4} - \frac{3}{2n} \right) \leqslant cn \left( \frac{1}{c}+ \frac{3}{4} \right) \leqslant cn
\end{WithArrows}\]

Siamo partiti dall'assunzione che \(j\) assuma equiprobabilisticamente tutti i valori compresi fra 1 ed \(n\).
Se questa assunzione non fosse vera allora i nostri calcoli non avrebbero alcun fondamento.
Forziamo quindi l'assunzione iniziale scegliendo un valore casuale come perno (\(A[\random{start, end}]\)) a differenza di prima dove prendevamo il primo valore (\(A[start]\)).
Questo accorgimento vale anche per \quickSort.
La complessità nel caso medio diventa quindi \(\Omicron(n)\).

\subsection{Selezione deterministica}

Supponiamo di avere un algoritmo \enquote{black box} che mi ritorni un
valore che dista al più \(\frac{3}{10} n\) dal mediano nell'ordinamento.

\paragraph{Implementazione}
L'implementazione dell'algoritmo consiste nei seguenti passaggi:

\begin{itemize}
	\item suddividi i valori in gruppi di 5. Chiameremo l'\(i\)-esimo gruppo \(S_i\), con \(i \in [1, \ceil{\nicefrac{n}{5}} ]\)
	\item trova il mediano \(M_i\) di ogni gruppo \(S_i\)
	\item tramite una chiamata ricorsiva, trova il mediano \(m\) dei mediani
	\([M_1, M_2, \ldots, M_{ \ceil{\nicefrac{n}{5}} }]\)
	\item usa \(m\) come pivot e richiama l'algoritmo ricorsivamente sul vettore opportuno, come nella \selection randomizzata
	\item quando la dimensione scende sotto una certa dimensione, possiamo utilizzare un algoritmo di ordinamento per trovare il mediano
\end{itemize}

\NoCaptionOfAlgo
\begin{algorithm}[H]
\caption{inserisci didascalia}

\prototype{\Item{} \select{\Item{} \(A\), \Int \(\First\), \Int \(\Last\), \Int \(k\)}}{
	\tcp{se la dimensione è inferiore ad una soglia (10), ordina il vettore e}
	\tcp{restituisci il \(k\)-esimo  elemento di \(A[\First][\Last]\)}
	\If{\(\Last - \First + 1 \leqslant 10\)}{
		\insertionSort{\(A\), \(\First\), \(\Last\)} \tcp{Versione con indici inizio/fine}

		\BlankLine
		\Return \(A[\First+k-1]\)\;
	}

	\BlankLine
	\tcp{divide \(A\) in \(\ceil{n/5}\) sottovettori di dim. \(5\) e ne calcola la mediana}
	\(M\) \Assign \new \Array{\Int}[1][\ceil{\nicefrac{n}{5}}]\;
	\lnl{select:calcolo-mediana}%
	\From{\(i \Assign 1\) \DownTo \(\ceil{\nicefrac{n}{5}}\)}{
		\(M[i] \Assign \medianFive(A, \First + (i-1) \cdot 5, \Last)\)
	}

	\BlankLine
	\tcp{individua la mediana delle mediane e usala come perno}
	\lnl{select:ricorsiva-prima}%
	\Item \(m \Assign \select \left(M, 1, \ceil{\nicefrac{n}{5}}, \ceil{ \frac{ \ceil{\nicefrac{n}{5}} }{2} } \right)\)\;

	\BlankLine
	\Int \(j \Assign \pivot{A, \First, \Last, m}\) \tcp{Versione con \(m\) in input}

	\BlankLine
	\tcp{calcola l'indice \(q\) di \(m\) in \([\First \ldots \Last]\)}
	\Int \(q \Assign j - \First + 1\)\;

	\BlankLine
	\tcp{confronta \(q\) con l'indice cercato e ritorna il valore conseguente}
	\uIf{\(q \Equal k\)}{
		\Return \(m\)\;
	}
	\uElseIf{\(q < k\)}{
		\lnl{select:ricorsiva-seconda}%
		\Return \select{\(A\), \(\First\), \(q-1\), k}\;
	}
	\Else{
		\lnlset{select:ricorsiva-terza}{3}%
		\Return \select{\(A\), \(q+1\), \(\Last\), \(k-q\)}\;
	}
}

\BlankLine
\tcp{calcola la mediana fra 5 elementi}
\prototype{\Int \medianFive}{

	\BlankLine
	\tcp{Restituisco la mediana}
	\Return \(m\)
}

\end{algorithm}
\RestoreCaptionOfAlgo

\paragraph{Analisi della complessità}


\begin{itemize}
	\item[\circled{\ref{select:calcolo-mediana}}]
	il calcolo dei mediani \(M[]\) richiede al più \(6\, \ceil{n/5}\)  confronti;

	\item[\circled{\ref{select:ricorsiva-prima}}]
	la prima chiamata ricorsiva dell'algoritmo \select viene effettuata su \(\lceil n/5 \rceil\) elementi;

	\item[\circled{\ref{select:ricorsiva-seconda}}]
	la seconda chiamata ricorsiva dell'algoritmo \select viene effettuata al massimo  su \(7 \frac{n}{10}\) elementi (esattamente \(n - 3\, \ceil{ \frac{ \ceil{\nicefrac{n}{5}} }{2} }\)).
\end{itemize}

L'algoritmo \select esegue nel caso pessimo \(O(n)\) confronti:
\begin{equation*}\textstyle
T(n) = T\left( \frac{n}{5} \right) + T\left( 7 \frac{n}{10} \right) + \frac{11}{5} n = \Omicron(n)
\end{equation*}

\ifsubfile
\end{document}
\fi
